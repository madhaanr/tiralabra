\documentclass[11pt,a4paper]{article}
\usepackage[finnish]{babel}
\usepackage[utf8]{inputenc}


\author{Marko Haanranta}
\date{\today}
\title{Viikkoraportti6}

\begin{document}
\maketitle
\section{Mitä opin tällä viikolla}
Tällä viikolla viimeistelin fibonaccin keon delete metodin ja yhdessä ohjaajien kanssa debuggasimme AVL-puun delete metodin kuntoon. Lisäksi osallistuin toiseen koodi katselmointiin. Osallistuin myös viralliseen demo tilaisuuteen jossa esittelin omaa työtäni. 
\section{Mikä jäi epäselväksi?}
Ei epäselvyyksiä, koska työ on valmis.
\section{Työn edistyminen}
pe 7.6.
AVL-puun testailua. Käännöt on nyt testattu melko hyvin ja näyttävät toimivan odotetusti kun puussa
on pieni määrä nodeja. Todella suurella node määrällä testailu ei ole mahdollista.
Deleten debuggaus jatkuu.
ma 10.6.
Fibonaccin keon consolidate ja removemin toimivat lopultakin oikein. Seuraavaksi korjaan avl-keon deleten ja
sitten kaikkien operaatioiden pitäisi olla toteutettu. Keskiviikon ja perjantain ohjauksissa voi tietysti tulla vielä
uusia juttuja.
ke 12.6.
Koodi katselmointiin valmistautumista. AVL-puun deleten korjailua. Debuggasimme deleteä noin kaksi tuntia
ohjaajien kanssa ja nyt se toimii. Tietorakenteet kurssin materiaalin pseudokoodi ei ottanut huomioon paria
erikoistapausta joten koodista puuttui muutama tarvittava kääntö. Yleistä refactorointia ja testailua.
to 13.6.
Demoon valmistautumista. Koodin refactorointia ja testailua. Dokumenttien kirjoittelua.
Demo tilaisuuteen osallistuminen.
pe 14.6.
Tietorakenne vertailua ja dokumenttien kirjoittelua.
la 15.6.
Dokumenttien kirjoittelua ja tietorakenteiden refactorointia.
su 16.6.
Dokumenttien viimeistelyä ja tietorakenteiden refactorointia.
\section{Mitä teen seuraavaksi?}
Pidän lomaa.
\end{document}